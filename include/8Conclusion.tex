% CREATED BY MAGNUS GUSTAVER, 2020
\chapter{Conclusion}
%Based on the results, it can be concluded that the proposed drone application configuration has the potential to generate substantial cost savings and enhance the overall efficiency of AstaZero. One of the key advantages of this project is the development of standardized testing scenes that can be automated and replicated for future tests, making it easier to compare safety assistance of different vehicles based on the captured footage. By utilizing drones, the possibility of human errors during filming is eliminated, and the need for human intervention is significantly reduced, leading to an increased lifespan of the drones. Overall, this innovative approach not only improves the accuracy of testing but also ensures a more sustainable and reliable testing process for AstaZero. The proposed drone-app-configuration has the potential to improve the efficiency and accuracy of the testing process at AstaZero while also providing significant cost savings, which can be beneficial.

% The system was able to capture footage that met the requirements of Euro NCAP testing standards, suggesting that it could be a valuable tool for manufacturers looking to improve their testing of car safety assistance processes. \newline

The drone application developed for Euro NCAP test documentation has the potential to save costs and enhance efficiency at AstaZero by reducing human errors and interventions. The approach presented in this report improves testing accuracy and sustainability, making the test process more reliable and streamlined.
\\ \\
AstaZero expressed their desire to find a solution to address the drones limited battery life. This desire was mainly in relation to the battery's capacity being unable to document several tests back to back. The solution was to add the ''Dry Run'' button since documenting the unofficial rehearsal tests were redundant. This lowers the amount of flight time which saves battery. 
\\ \\
The application could potentially utilize other means of saving battery, such as using the zoom function instead of visiting every waypoint in a mission. This means that the drone could fly on a higher altitude with a lower speed and mostly utilizing the camera to capture the desired angles. However, since the ISO22133 protocol lacks the capability of transmitting information about zoom or the location of other objects, this is hard to achieve.
\\ \\
However, some technical challenges need to be addressed to improve the performance of the system. The application is not capable of automatically adjusting the camera-gimbal for capturing the relevant video frames. This means that an operator still needs to control the gimbal, but not the actual flying. This leads to better performance and smoother video capturing compared to how the tests were captured before, but does not remove the need for a person operating the drone.
\\ \\
In conclusion, many wishes and subtasks presented has been addressed and solved, making the application a contender instead of manual flight and operation of the drone. Even though some challenges still are unsolved, this project has demonstrated the ability to use drones for automating documentation of Euro NCAP tests and has made an significant contribution the existing code base at AstaZero. 
\\ \\
The team's recommendation to AstaZero is to publish the continuation of this project in next year's Bachelor thesis repertory to guarantee the finalization of the application and the ATOS module, if AstaZero does not intend on completing it themselves. Although the remaining work to attain complete functionality may not be vast, there is always room for improvement.