% CREATED BY MAGNUS GUSTAVER, 2020
\chapter{Conclusion}


%Based on the results, it can be concluded that the proposed drone application configuration has the potential to generate substantial cost savings and enhance the overall efficiency of AstaZero. One of the key advantages of this project is the development of standardized testing scenes that can be automated and replicated for future tests, making it easier to compare safety assistance of different vehicles based on the captured footage. By utilizing drones, the possibility of human errors during filming is eliminated, and the need for human intervention is significantly reduced, leading to an increased lifespan of the drones. Overall, this innovative approach not only improves the accuracy of testing but also ensures a more sustainable and reliable testing process for AstaZero. The proposed drone-app-configuration has the potential to improve the efficiency and accuracy of the testing process at AstaZero while also providing significant cost savings, which can be beneficial.

The proposed drone application configuration has the potential to generate substantial cost savings and enhance the overall efficiency of AstaZero. The use of drones eliminates the possibility of human errors during filming and significantly reduces the need for human intervention, leading to a lower cost for the company. This innovative approach not only improves the accuracy of testing but also ensures a more sustainable and reliable testing process for AstaZero. Therefore, the proposed drone application has the potential to improve the efficiency and accuracy of the testing process at AstaZero while also providing significant cost savings. \newline

In conclusion, this project has been partly successful with a system developed for autonomous flight for a set trajectory but lacks the object tracking implementation. The project can still be utilized to film EuroNCAP tests that does not require a camera adjustment, such as the test the team chose to focus on. So the system is able to capture high-quality video footage of car safety tests with minimal human workload, representing a significant improvement over traditional methods of capturing video footage during these tests. \newline

% The system was able to capture footage that met the requirements of Euro NCAP testing standards, suggesting that it could be a valuable tool for manufacturers looking to improve their testing of car safety assistance processes. \newline

However, some technical challenges need to be addressed to improve the performance of the system. For example, there were some issues related to the drone that has not been fully addressed, which are limitations related to the battery life of the drones and the ability to use multiple drones at once. Together with the contact at AstaZero, the team provided a way to save battery by using ''dry runs''. This lowers the amount of flight time which saves battery. The application could potentially utilize other means of saving battery, such as using the zoom function instead of flying. However, since the ISO22133 protocol is not able to provide the drone with information about zoom or the location of other objects, it is hard to achieve.\newline

Despite these challenges, this project has demonstrated the potential of using drones for automating Euro NCAP tests, which could ultimately lead to safer cars on the road and fewer accidents. Future research could investigate ways to improve the test by the ability to run every existing Euro NCAP test and as stated earlier use mulitply drones at once to get an even better documentation of the test.\newline

Overall, this project has made an important contribution to the field of automotive safety testing by demonstrating how drones can be used to automate video documentation processes. The findings suggest that drone-based systems have significant potential as a tool for improving car safety testing processes and reducing costs for manufacturers.