\chapter{Future work}
The current system has limited capabilities and there are plenty of opportunities for future work to improve and expand upon this technology.
\\ \\
One area which requires further development is the task not completed, the object detection and tracking algorithm. Essentially, there are two approaches to this challenge. The first approach involves parsing the YUV data correctly and displaying it on the screen while finding a way to still being able to send it to the object detection algorithm.
The second approach entails not utilizing the YUV callback at all and instead attempting to implement a different callback or receiver in order to display the camera view on the screen. Further testing and research about the capabilities and limitations of the DJI SDK is needed to determine the best course of action.
\\ \\
As detailed in Sec.~\ref{sec:existing_code_base}, the selection of compatible object detection packages was limited by the outdated Android SDK version. Nonetheless, TensorFlow Lite was identified as a promising solution owing to its fast response time and acceptable accuracy levels. Our experiments revealed that the camera's capability to track the selected object was fundamental, thereby emphasizing the significance of prioritizing response speed over accuracy.
\\ \\
In light of this observation, any changes of model should focus on maximizing speed rather than enhancing accuracy by incorporating a bulkier and more sophisticated model. This strategy will ensure that the application remains agile and effective on any Android mobile device, while concurrently fulfilling its primary goal of tracking objects. Currently, the project uses a single drone to capture video footage of Euro NCAP tests. While this provides valuable information, it is limited by the fact that it can only capture footage from one angle at a time. By using three drones instead of one, it would be possible to capture footage from multiple angles simultaneously, providing a more comprehensive view of the test or if the test instructions would change.
\\ \\
Theoretically, a test could be documented using any number of drones, where each drone is linked to a unique instance of the application running on separate mobile devices. The problem lies within the ATOS module that the PSU team was responsible for, which currently is not designed to create different trajectories for three different drones. Since this project only focused on the mobile application, it lied outside the scope of this project but could be a subject for future work.