\chapter{Future work}
\todo{Det finns ganska mycket att skriva här. Jag började och så får gärna vem som helst fylla på.}
This project has demonstrated the capability to connect in-house developed software, in this case ATOS with DJI SDK to make an autonomous documentation of an Euro NCAP test by use of a drone. While the current system has some limitations, there are plenty of opportunities for future work to improve and expand upon this technology.
\\

\todo{Tycker ni detta stämmer? Tänkte på att object detection känns inte hundra just nu i och med att vi inte kan använda så bra algoritmer} One area is in the devolopment to implement an even more advanced application which could be compatible with an even more resource heavy object detection algorithm. For example, deep learning techniques could be used to automatically detect and classify objects such as other vehicles or pedestrians. 
\\

It is important to acknowledge the ethical and safety considerations associated with employing drones in safety testing. One such concern is the potential breach of privacy that may arise if drones are employed to record video footage of multiple tests being conducted simultaneously, or in areas where people may not anticipate surveillance. Moreover, there exists a plausible risk to safety in the event of a drone malfunction or collision with a foreign object.\todo{Kopplade vi bort object detection? Minns inte riktigt, ta upp detta med gruppen} during a test.
\\
\newline 
The fact is that AstaZero currently has three drones available. Another area, one such area is the ability to use multiple drones at once. Currently, the system developed in this project uses a single drone to capture video footage of Euro NCAP tests. While this provides valuable information, it is limited by the fact that it can only capture footage from one angle at a time. By using three drones instead of one, it would be possible to capture footage from multiple angles simultaneously, providing a more comprehensive view of the test  or if the test instructions would change.
\\
 
\todo{slutkläm | löser / Leo | ej tar du min text /Teddy | nejnej / Leo}
In conclusion, this project has demonstrated that using drones for automated video documentation in Euro NCAP tests is a promising technology with many opportunities for future work. By improving image processing algorithms, designing better drones, exploring new applications, and addressing ethical and safety concerns, this technology has the potential to revolutionize safety testing and inspection in the automotive industry.

Numerous facets of the drone application and the ATOS-module  present opportunities for improvement. Nonetheless, the foundation established through this project is robust and can be extended with ease, provided that the next group acquires familiarity with the modules. The team's recommendation to AstaZero is to publish the continuation of this project in next year's Bachelor thesis repertory to guarantee the finalization of the application and the ATOS module, if AstaZero does not intend on completing it themselves. Although the remaining work to attain complete functionality may not be vast, there is always room for improvement.
