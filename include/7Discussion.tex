\chapter{Discussion} \label{sec:discussion}

A fully developed application could be substantially beneficial for AstaZero while performing the assessments included in the Euro NCAP portfolio. Implementation of a fully finished composition of the application and the ATOS-module would guarantee accurate, reliable, and most of all, consistent footage of the various test scenarios. The standardized approach for footage-capturing would also simplify the comparison between vehicles that are tested, which of course is of essence when grading them. Furthermore, as the cost-benefit analysis concluded in Sec.~\ref{res:COSTBENEFIT}, the implementation is also financially advantageous and beneficial in terms of time-consumption for AstaZero. 
\\ \\
The sheer scale and complexity of this project have posed significant challenges that required numerous problem-solving endeavors throughout the process. As previously mentioned, this project has been conducted in collaboration with six students from Penn-State University in State College, PA. The six hour time-difference and the diverse schedules between the groups and the members within the group, has in some instances been challenging. Without the division of labor between the two groups from the different universities, this would likely have been a larger obstacle. However, with a project of this nature, dividing the tasks led to a limited influence on the other group's output, which implicates the need of holding both groups accountable in case of any failure to fulfill their designated responsibilities, which unfortunately was the case. 
\\ \\
The ATOS-module which was developed by the PSU-team was not finished on time for their graduation which ultimately led to that the strictly necessary ATOS-module could not be implemented in the fully-fledged package. Due to the Chalmers team's limited experience with the ATOS-module, the team could not simply continue where the Penn State-team left off. As previously mentioned, the ATOS-module sends the calculated trajectories for the drones, making it completely invaluable for the success of the project. The fact that this was not yet finished by the end of the project made it impossible to deliver a fully finished and integrated application. However, the Chalmers team has utilized other parts of ATOS to create static trajectories which are not calculated relative to a object. This made it possible to test the application developed by the team and does in fact simulate how it was intended to work, all thanks to the ISO22133 standard. The standard implicates that all data transmitted to the drone over the protocol will always be in the same format, making it easy to switch to the module PSU was responsible for to deliver the trajectories if it was to be finished, either by another project or by employees at AstaZero.
\\ \\
Moreover, the team encountered multiple challenges in their efforts to obtain accurate footage of the test. Initially, the application utilized the ActiveTrackOperator~\cite{DJIDJIActiveTrackOperator} from the DJI SDK and several weeks were spent attempting to incorporate this package into the application. However, it was later discovered that the WaypointMission and ActiveTrackOperator packages were incompatible and could not be run simultaneously. More research should have been conducted before committing to a singular solution, as this setback impeded the progress of the object detection development for several weeks as alternative approaches were explored.
\\ \\
Additionally, as the entire process relies on a mobile device with limited computational power, the application was restricted to only employ a streamlined and lightweight image recognition package that demands minimal processing power. Although the Samsung A13 performs well in certain areas, it cannot compete with a laptop or workstation, which can support more intensive software. If AstaZero was not satisfied with the performance of the model, another model could be used instead. There are many models within TensorFlow Lite which are trained on the same COCO dataset as mentioned in \ref{sec:image_rec}. Some models are even faster but not as accurate as the model used in this project.
\\ \\
As previously mentioned object tracking was not accomplished due to\todo{tog bort a} multiple reasons. Firstly, there was limited prior knowledge in object detection and no experience with object tracking. Secondly, as stated previously, not enough research was conducted on the compatibility of different DJI SDK components, especially the ActiveTrackMission in relation to WaypointMission. Thirdly, since the application that was provided by AstaZero runs on an old version of the Android SDK the options for object tracking packages were slim as mentioned in Sec.~\ref{sec:existing_code_base}. Lastly, due to an unfortunate accident where the drone crashed into a wall, it became impossible to continue testing new features. To be able to test different parts of the application, the mobile device needs to be connected to the remote controller which also needs a connection to the drone. Without an active connection, the application will not run and contributed to the reason why object detection was not completed on the drone and why object tracking still has not been implemented.
\\ \\
It is important to acknowledge the ethical and safety considerations associated with employing drones in safety testing. One such concern is the potential breach of privacy that may arise if drones are employed to record video footage of multiple tests being conducted simultaneously, or in areas where people may not anticipate surveillance. Moreover, there exists a plausible risk to safety in the event of a drone malfunction or collision with a foreign object during a test.
\\ \\
All in all, the Chalmers team is satisfied with their accomplishments, even though a fully functional solution can not be delivered to AstaZero. The application can serve as a foundation for future work by either AstaZero or another Bachelor's thesis project, adding even more functionality to simplify the documentation of Euro NCAP tests using drones. 
