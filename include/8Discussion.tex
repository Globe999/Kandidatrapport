\chapter{Discussion}

As previously stated, a fully developed application could be substantially beneficial for AstaZero while performing the assessments included in the Euro NCAP portfolio. Implementation of a fully finished composition of the application and the ATOS-module would guarantee accurate, reliable, and most of all, consistent footage of the various test scenarios. The standardized approach for the footage-capturing would also simplify the comparison between the vehicles that are tested, which of course is of essence when grading them. Furthermore, as the cost-benefit analysis concluded \ref{Cost-benefit} the implementation is also advantageous financially and in terms of time-consumption for AstaZero. 
\\

The sheer scale and complexity of this project have posed significant challenges that required numerous problem-solving endeavors throughout the process.

\todo{FORTSÄTT SKRIVA OM VILKA (tekniska?) PROBLEM VI STÖTT PÅ}
\\


As previously mentioned, this project has been conducted in collaboration with six students from Penn-State University in State College, PA. The six hour time-difference and the diverse schedules between the groups and the members within the group, has in some instances posed a challenge. Without the division of labor between the two groups from the different universities, this would likely have been a larger obstacle. However, with a project of this nature, dividing the tasks led to a limited influence on the other group's output, which implicates the need of holding both groups accountable in case of any failure to fulfill their designated responsibilities, which unfortunately was the case. 
\\

The ATOS-module which was developed by the PSU-team was not finished on time for their graduation which ultimately led to that the strictly necessary ATOS-module could not be implemented in the fully-fledged package. Due to the Chalmers team's limited experience with the ATOS-module, the team could not simply continue where the PSU-team left off. As previously mentioned, the ATOS-module sends the calculated trajectories for the drones, making it completely invaluable for the success of the project. The fact that this was not yet finished by the end of the project made it impossible to deliver a fully finished and integrated application. However, the Chalmers team has written their own, provisional, trajectory files in order to ensure that the drone-application works as intended, in terms of overall drove-movement. 
\\

Moreover, the Chalmers team faced several difficulties regarding ensuring that the camera captured the correct footage. The cause of action was to create a sufficient object tracking program. As stated in chapter \ref{} \todo{HÄNVISA TILL DET VI SKRIVIT OM ATT ACTIVE TRACK INTE FUNKAR SAMTIDIGT MED WAYPOINTMISSION} the drone was programmed to use the DJI SDK's ActiveTrackOperator and weeks were spent trying to implement this with the existing software that had already been written, only to realize that the ActiveTrackOperator could not be used simultaneously with the DJI SDK's WaypointMission. Furthermore, since the whole process is reliant on a cellphone with limited processing power, the team was limited to a lightweight program in terms of required processing. 

All in all, the Chalmers team is satisfied with their accomplishments, even though a fully functional solution can not be provided for AstaZero. With the limited experience in software development that the group had prior to taking on this project, the output is a functional application that performs as expected.


\todo{Förklara varför image recognition stanna där det gjorde/JG}


