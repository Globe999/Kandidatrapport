% CREATED BY MAGNUS GUSTAVER, 2020
Automate video documentation of Euro NCAP tests by use of drones\\
EENX16-23-26\\
Department of Electrical Engineering\\
Chalmers University of Technology \setlength{\parskip}{0.5cm}

\thispagestyle{empty}			% Supress header 
\setlength{\parskip}{0pt plus 1.0pt}
\section*{Abstract}
The objective of this project was to create an automated system for video documentation of Euro NCAP tests utilizing a camera-equipped drone. Specifically, the intention was to showcase the potential of using an autonomous drone to capture video footage of a car, or another vehicle undergoing a Euro NCAP test on the AstaZero testing grounds. The report details how an autonomous drone can be used to film these tests safely and effectively. 
\\

To achieve this objective, an Android drone application was developed which governs the drone's movement. The aim was to incorporate an object tracking algorithm, enabling the cameras on the drones to maintain direct focus on the vehicle being tested. This methodology aims to fortify the system's durability and ensure the camera records high-quality footage throughout the entire testing process. The tracking algorithm was however not finished on time resulting in the application only incorporating the object detection part.  
\\

The report also describes the challenges associated with using autonomous drones for filming, such as working with predetermined paths, staying connected to a control center and pulling live video feed to process it. To address these problems, several tests and adjustments were made to the drone's software. 
\\

Overall, this project represents an important stride forward in the automation of Euro NCAP tests and has the potential to significantly improve the efficiency and accuracy of these tests in the future. 


% KEYWORDS (MAXIMUM 10 WORDS)
\vfill
\textbf{Keywords:} Autonomous drone, video documentation, Euro NCAP test, AstaZero, ATOS

		% Create empty back of side
% \thispagestyle{empty}
\mbox{}