\thispagestyle{plain}			% Supress header 
\setlength{\parskip}{0pt plus 1.0pt}
\section*{Sammanfattning}
Syftet med detta projekt var att skapa ett automatiserat system för videodokumentation av Euro NCAP-tester med hjälp av en kamerautrustad drönare. Mer specifikt är avsikten att utveckla en demonstration som visar en autonom drönares förmåga att följa och filma en bil eller ett annat fordon som genomgår ett Euro NCAP-test på företaget AstaZeros testområde. Genom rapporten beskrivs det hur en autonom drönare kan användas för att filma dessa tester på ett säkert och effektivt sätt.
\\

För att uppnå detta mål utvecklades en Android-drönarapplikation som styr drönarens förflyttning, samt en objektföljningsalgoritm, vilket skulle möjliggöra att kameran på drönaren kan bibehålla fokus på fordonet som testas. Denna metod syftar till att stärka systemets robusthet och säkerställa att kameran fångar högkvalitativt videomaterial under hela testprocessen. Algoritmen färdigställdes dock inte i tid, applikationen innehåller därför endast objektigenkänning.
\\

Rapporten beskriver också utmaningarna med att använda autonoma drönare för att filma, såsom att arbeta med förutbestämda rutter, bibehålla anslutning till ett kontrollcenter och bearbeta kameraströmmen. För att lösa dessa problem genomfördes flera tester och justeringar av drönaren och dess programvara.
\\

Sammanfattningsvis är projektet ett viktigt steg framåt inom automatiseringen av Euro NCAP-tester och har potentialen att avsevärt förbättra effektiviteten och noggrannheten hos dessa tester i framtiden.


% KEYWORDS (MAXIMUM 10 WORDS)
\vfill
\textbf{Nyckelord:} Autonom drönare, videodokumentering, Euro NCAP test, AstaZero, ATOS

		% Create empty back of side
\mbox{}