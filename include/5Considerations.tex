\chapter{Considerations}
Drones are generally considered to be environmentally friendly, there are however certain risks that come with drones and vehicle testing.

\section{Ethical aspects}
Although the project did not have a direct impact on the general public's life in an ethical aspect or in a socio-economic point of view, there were still some important considerations to keep in mind. At the time, AstaZero performed manual calculations and transmissions of paths, requiring man-hours. Upon completion of the project, the work-load was likely to decrease, leading to a shift in labor distribution. Although the elimination of manual adjustments could lead to a reduced demand for staff, it was deemed to be a non-feasible outcome. Contrarily, the development of the application would serve as a foundation for further software development, implying an increase in labor demand. Additionally, there would be a need for maintenance and quality control, creating even more job opportunities.
\\ \\
In terms of ethical aspects, there were a few more things to keep in mind. The project included video documentation, and it was crucial to abide by the laws and relevant guidelines in this regard. Furthermore, during the NCAP tests, there could be vehicles and classified information that should not be filmed. Therefore, careful consideration had to be given to the implementation of the drone app to ensure that the captured footage was limited only to the desired vehicle and did not include any sensitive or restricted information.

\section{Environmental aspects}
As with many electric appliances or vehicles, most drones are powered by lithium batteries. The production of lithium batteries comes with many environmental problems, such as a lot of water being used when mining for lithium, which dries surrounding land and severely harms the local ecosystem~\cite{BauerSophie2020Explainer:Industry}. Potential leakage that has been known to occur in these mines can poison the surrounding wild life~\cite{AmitKatwala2018TheAddiction}. However, the implementation of drones is not unique to our project, our goal is to automate the drones instead of manually flying them. To truly consider environmental aspects of the project, one should consider alternatives to drones capturing the footage from the tests. Since the implementation of drones, footage from a camera moving alongside a car is possible. This footage can probably provide better material to analyse a car's performance in certain tests. Alternatives to having a drone move alongside a car being tested include filming from some other vehicle. Few vehicles are however as light as drones, and if the battery is charged with electricity from renewable sources, the environmental impact of using the drone can be considered almost negligible. 


\section{Laws and rules} \label{Laws and rules}
The operation of drones, is subject to varying rules and regulations across different countries. In Ch.~\ref{chap:problem statement}, the problem statement declares AstaZero's objective of expanding drone flight capabilities beyond its Swedish test track. However, accomplishing this goal presents a formidable challenge due to the regulatory framework. Considering that the project pertains to European testing, the focus is solely on complying to regulations within Europe. The increasing expansion  of drones necessitates constant adaptation of regulations, further compounded by the divergent rules prevailing in each country. Consequently, conducting comprehensive research on the regulatory requirements becomes vital when contemplating tests in countries other than Sweden. Additionally, it is essential to note some general rules applicable to flying drones, particularly within the context of private property, as AstaZero is unlikely to undertake Euro NCAP tests outside a designated test track.

\subsection{Overview of Drone Regulations}
The operation of drones is governed by a huge number of regulations that differ from one country to another. These regulations are established at the national level and may be supplemented by international standards, such as those implemented by the European Union Aviation Safety Agency (EASA)~\cite{PublicationofficeoftheEuropeanUnion2019CommissionSystems}, to ensure consistency across Europe.

\subsection{Dynamic Nature of Regulations}
The increasing popularity of drones has led to a continuous evolution of regulatory frameworks. As a result, regulations related to drones undergo frequent updates, making it imperative for all parties involved to remain informed about the latest requirements. Moreover, due to the divergent regulations across countries, it is crucial to conduct thorough research to ensure conformity when planning drone tests outside Sweden. It is worth noting that although AstaZero primarily focuses on conducting tests on private property, certain general rules should still be taken into consideration.

\subsection{General Rules for Flying Drones on Private Property}
When operating a drone on private property, the following rules should be complied with according to EASA~\cite{PublicationofficeoftheEuropeanUnion2019CommissionSystems} and Transportstyrelsen~\cite{TransportstyrelsenDronesAircraft}:\\
\begin{itemize}
    \item 
Registration: Prior to flight, it is necessary to register the drone on the official website of the transport department or the relevant regulatory authority.
\item Online Course Completion: Drone operators are required to complete an online course covering the essential knowledge and guidelines for safe and responsible drone operation.
\item Altitude Limitations: Drone flights should not exceed a maximum altitude of 120 m or 400 feet, ensuring compliance with airspace regulations and safety protocols.
\item Minimum Pilot Age: The pilot operating the drone must be at least 16 years of age, demonstrating maturity and proficiency in UAV handling.
By adhering to these general rules, AstaZero can ensure responsible drone operations while considering the specific regulations of the country where tests are conducted.
\end{itemize}

\bigskip
In conclusion, the operation of drones is governed by diverse and ever-changing regulations across different countries. AstaZero's ambition to expand drone flight capabilities beyond its Swedish test track presents a significant challenge due to the regulatory framework. With a focus on European testing, compliance with regulations within Europe is of utmost importance. The increasing popularity of drones necessitates continuous adaptation of regulations, and comprehensive research is crucial when planning tests in countries other than Sweden. Observing general rules, including registration, online course completion, altitude limitations, and minimum pilot age, ensures responsible drone operations. By considering these factors, AstaZero can navigate the complex regulatory landscape and conduct successful drone tests.




